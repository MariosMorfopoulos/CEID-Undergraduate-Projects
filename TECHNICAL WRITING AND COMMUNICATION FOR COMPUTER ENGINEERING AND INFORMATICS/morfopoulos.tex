\documentclass[11pt,a4paper,twoside,english,greek]{article}
\usepackage{graphicx}
\usepackage{epstopdf}
\usepackage{indentfirst}
\usepackage{verbatim}
\usepackage{amsmath}
\usepackage{amsthm}
\usepackage{amssymb}
\usepackage{latexsym}
\bibliographystyle{IEEEtran}
\usepackage{hyphenat}
\usepackage{makeidx}
\usepackage{algpseudocode}
\usepackage{algorithm}
\usepackage[hyphens]{url}
\usepackage[hyphenbreaks]{breakurl}
\usepackage{enumitem}
\usepackage{xspace}
\usepackage{booktabs}
\usepackage{multirow}
\usepackage{subfig}
\usepackage{tabularx}
\usepackage{listings}
\usepackage{xcolor}
\usepackage{bbding}
\usepackage{footmisc}
\usepackage[greek,english]{babel}
\usepackage[inner=3cm,outer=3cm]{geometry}
\renewcommand{\baselinestretch}{1.2} 

\begin{document}
\title{Συγγραφή και παρουσίαση τεχνικών κειμένων}
\author{{\LARGE Μορφόπουλος Μάριος - AM: 1058102}}
\maketitle


\newpage
\section*{Επιλογή άρθρων}
\vspace{12mm}
{\Large {ONOMATEΠΩΝΥΜΟ: ΜΟΡΦΟΠΟΥΛΟΣ ΜΑΡΙΟΣ ΑΛΕΞΑΝΔΡΟΣ
\newline \newline \selectlanguage{english}	SHA-1: be0b2867410966cd38961a720c54fe2b98635aeb 
\newline
\newline \newline \selectlanguage{greek} Διακριτά ψηφία: \selectlanguage{english}  b,e,0,2,8,6 \newline \newline
}

\section*{Ανασκόπηση άρθρων}
\vspace{3 mm}
\setlength{\parindent}{1cm} Συχνά συμβαίνει σε πολλές επιχειρήσεις να υπάρχει μια αδύναμη σχέση μεταξύ των απαιτήσεων λογισμικού και των δοκίμων του. Ο μηχανικός που καθορίζει τις απαιτήσεις του προϊόντος λογισμικού συχνά δεν είναι διαθέσιμος για τον ελεγκτή – δοκιμαστή του προϊόντος, ενώ κανονικά πρέπει να αποτελεί τον πιο στενό του σύμμαχο. Για το λόγο αυτό, η απαιτούμενη ευθυγράμμιση μεταξύ των μηχανικών που καθορίζουν τις προδιαγραφές λειτουργίας των προϊόντων λογισμικού για τις επιχειρήσεις και των αντίστοιχων δοκιμαστών στην πράξη, αποτελεί μια σημαντική πρόκληση για τις εταιρείες και αποτελεί μια σημαντική προϋπόθεση για την δημιουργία ενός αποτελεσματικού συστήματος λογισμικού.
Πολλές φορές η λειτουργία σε ένα τομέα όπως αυτό της Πληροφορικής, που έχει διαρκώς μεταβαλλόμενες απαιτήσεις και συντόμους κύκλους παράδοσης, επιδεινώνει τις δυσκολίες για μια έγκυρη παράδοση. Η  κατάσταση αυτή καθίσταται ολοένα και πιο συχνή, αφού οι εταιρείες χρησιμοποιούν την συνεχή παράδοση προϊόντων  για να συμβαδίσουν με τις διαρκώς μεταβαλλόμενες απαιτήσεις της αγοράς.
 
\setlength{\parindent}{1cm} Η χρήση του διαδικτύου των πράγματων έχει εξαπλωθεί τα τελευταία χρόνια σε διάφορους τομείς όπως είναι η εκπαίδευση η υγειονομική περίθαλψη, οι μεταφορείς η διαχείριση της κυκλοφορίας στις μεγάλες πόλεις κ.λ.π. Εκατοντάδες νέες εταιρίες έχουν δημιουργηθεί σε αυτόν τον τομέα και επίσης οι περισσότερες μεγάλες εταιρίες έχουν ανακοινώσει ευρεία χρήση πλατφορμών IoT. Πάρα την φαινομενική ποικιλομορφία και τον μεγάλο αριθμό προμηθευτών υλικού για το διαδίκτυο των πράγματων, αναδύεται μια κοινή αρχιτεκτονική από άκρο σε άκρο για λύσεις IoT, με πολλά στοιχεία που είναι σχεδόν τα ίδια  σε όλα τα συστήματα. Κυρίαρχο ρόλο παίζουν οι αισθητήρες που παρέχουν ποικίλες πληροφορίες σχετικά με την παρακολουθούυμενη φυσική οντότητα.

\setlength{\parindent}{1cm} Οι προβλέψεις για τις εξελίξεις κατά τις επερχόμενες δεκαετίες λογισμικού  είναι: α) καθώς η αυτοματοποίηση αυξάνεται, αυξάνεται επίσης η ζήτηση για επαγγελματίες πληροφορικής, β) η πολυπλοκότητα του λογισμικού θα οδηγήσει αναπόφευκτα στην εξειδίκευση, γ)  η Τεχνητή Νοημοσύνη θα καταστεί από μονή της ένα πεδίο πρακτικής,  δ) η εποχή της εφεύρεσης θα προχωρήσει προς την εποχή της πρακτικής εξάσκησης ε) η κωδικοποίηση θα εξελιχθεί σε ένα τομέα εξειδίκευσης και επίσης θα χρησιμοποιηθεί στην μοντελοποίηση δεδομένων.

\setlength{\parindent}{1cm} Η τεχνολογία λογισμικού, η οποία αποτελεί ένα κλάδο της επιστήμης των υπολογιστών, έχει ως στόχο τη δημιουργία πρακτικών και οικονομικά αποδοτικών λύσεων σε προβλήματα επεξεργασίας πληροφοριών, εφαρμόζοντας κατά προτίμηση επιστημονικές γνώσεις και αναπτύσσοντας συστήματα λογισμικού. Στην σημερινή εποχή, οι προκλήσεις που  αντιμετωπίζουμε αναφορικά με την εκπαίδευση στην τεχνολογία λογισμικού είναι ποικίλες. Μεταξύ αυτών, οι πιο σημαντικές είναι η αναθεώρηση των προγραμμάτων σπουδών για πανεπιστημιακά προγράμματα κάλυψης των σημερινών αναγκών, όπως για παράδειγμα κατανεμημένα προσαρμοστικά συστήματα καθώς και η εύρεση τρόπων για εργασία με εκπαιδευτικούς σε άλλα πεδία για τη δημιουργία προτύπων σχεδίασης λογισμικού και εργαλείων διευκόλυνσης ατόμων που δεν είναι άριστα εκπαιδευμένοι και εξοικειωμένοι με τους υπολογιστές, να πετύχουν ποιοτικά αποτελέσματα.

\setlength{\parindent}{1cm} Οι προγραμματιστές χρησιμοποιούν συχνά γνώσεις από συλλογική διαδικτυακή δραστηριότητα προκειμένου να υποστηρίξουν αναπτυξιακά καθήκοντα και να συλλέξουν τα σχόλια των χρηστών. Ιστότοποι ερωτοαπαντήσεων έχουν καταστεί τα τελευταία χρόνια εξαιρετικά δημοφιλείς στους προγραμματιστές λογισμικού. Τυπικά, οι προγραμματιστές δημοσιεύουν ερωτήσεις σχετικές με αυτούς τους Ιστότοπους και ένας ή περισσότεροι από τους συμμετέχοντες μπορούν να παρέχουν απαντήσεις. Ουσιαστικά δηλαδή η απάντηση έχει ανατεθεί σε ένα πλήθος ατόμων. 

\setlength{\parindent}{1cm} Με την πάροδο των χρόνων, οι ιστότοποι ερωτήσεων και απαντήσεων εξελίχθηκαν, προκειμένου να κάνουν περισσότερα από το να απαντούν μόνο σε ερωτήσεις. Για να καθορίσουμε τον ρόλο τους στο σημερινό κύκλο ζωής ανάπτυξης λογισμικού, θα πρέπει να απαντήσουμε σε διάφορα ερωτήματα, όπως για παράδειγμα σε ποια θέματα είναι πιο χρήσιμο για το κοινό να λάβει απαντήσεις και επίσης ποια θέματα χρειάζονται περισσότερο χρόνο για να απαντηθούν. Στη συνέχεια παρέχονται διάφορες συστάσεις οι οποίες θα βελτιώσουν τις δυνατότητες των ιστότοπων που περιέχουν ερωτήσεις και απαντήσεις, προκειμένου αυτοί να ανταποκριθούν στην αυξανόμενη ζήτηση από τους προγραμματιστές. Μια από αυτές τις συστάσεις είναι η παροχή άμεσης ανατροφοδότησης (μελλοντικές εκδόσεις αυτών των ιστότοπων θα πρέπει να συμπεριλάβουν έναν μηχανισμό με τον οποίο οι προγραμματιστές θα μπορούν να αποκτούν άμεση ανατροφοδότηση από το πλήθος). Επίσης, μια δεύτερη σύσταση είναι ότι οι προγραμματιστές θα πρέπει να συνδέσουν τις δεσμεύσεις τους με τις συζητήσεις που διεξήχθησαν για να βοηθήσουν στην επίτευξη των τελικών κωδικοποιημένων λύσεων. Αυτές οι συζητήσεις θα μπορούσαν να συνεχίσουν να εξελίσσονται, οπότε και αν εντοπιστεί ένα σφάλμα στο δημοσιευμένο κώδικα, άλλοι προγραμματιστές θα μπορούσαν να βοηθήσουν στη παροχή μια ενημέρωσης ή διόρθωσης.

\setlength{\parindent}{1cm}Συνεχίζοντας πάνω στις γνωσιακές δεξιότητες που θα χρειασθεί ένας προγραμματιστής λογισμικού, θα πρέπει να αναφερθεί ότι οι περισσότερες από αυτές μπορούν να διδάσκονται έμμεσα σε ένα πανεπιστήμιο. Δεν αποτελεί έκπληξη το γεγονός ότι οι βασικές γνωσιακές δεξιότητες ενός ερευνητή περιλαμβάνουν όλα τα υπόλοιπα στοιχεία της ταξινόμησης του Bloom, που είναι: εφαρμογή, κατανόηση, ανάλυση, σύνθεση και αξιολόγηση. Ένας προγραμματιστής λογισμικού θα πρέπει να μπορεί να εφαρμόζει την θεωρητική του γνώση πάνω στις εκάστοτε συγκεκριμένες απαιτήσεις λογισμικού. Όσον αναφορά την κατανόηση, θα πρέπει να είναι σε θέση να ερμηνεύσει τις υπάρχουσες ακολουθίες κώδικα, να τις επεκτείνει ώστε να ταιριάζουν με τις νέες απαιτήσεις και αν είναι δυνατόν να τις επαναπροσδιορίσει, προκειμένου να μπορέσει να μειώσει το υπολογιστικό κόστος. Επίσης, πρέπει να έχει την δυνατότητα να μετατρέψει τις προδιάγραφες ενός προϊόντος σε κώδικα και στη συνέχεια να συνοψίσει τον κώδικα και να τον εξηγήσει σε συναδέλφους. Στις εργασίες ανάπτυξης λογισμικού απαιτούνται αναλυτικές δεξιότητες για τον εντοπισμό πιθανών αιτιών σφαλμάτων και προβλημάτων απόδοσης και επίσης δεξιότητες αναφορικά με τη σύγκριση ανταγωνιστικών προγραμμάτων λογισμικού για τον εντοπισμό των πιο κατάλληλων, ανάλογα με την εκάστοτε περίπτωση.

\bigbreak				
Τέλος, μπορούν να μελετηθούν και άλλες ενδιαφέρουσες πηγές \cite{Kersten2017},\cite{Broy2018},\cite{KlotinsUnterkalsteinerGorschek2018},\newline\cite{Booch2018},\cite{GouesJaspanOzkayaShawStolee2018},\cite{HodaSallehGrundy2018},\cite{LaporteMunozMirandaConnor2018},\cite{WilliamsMcGrawMigues2018},\cite{Holzmann2018},\cite{LeichtBlohmLeimeister2017}, για το θέμα {\selectlanguage{english} \bibliography {morfopoulos}}

\thispagestyle{empty}


\thispagestyle{empty}
\onecolumn
 \thispagestyle{empty}
\twocolumn
\thispagestyle{empty}
\vspace{6mm}
\begin{center}
{\Huge Μάριος Μορφόπουλος }
\newline

\includegraphics{fig.jpg}
\end{center}

\newpage
\vspace{12mm}
\bigbreak
O Μάριος Μορφόπουλος είναι ενεργός φοιτητής του τμήματος Μηχανικών Η/Υ και Πληροφορικής του Πανεπιστημίου Πατρών. Βρίσκεται στο 3ο έτος φοίτησης και τα ενδιαφέροντά του αναφορικά με την ειδικότητά του, είναι ο προγραμματισμός και ιδιαίτερα οι γλώσσες {\selectlanguage{english} C++, Java, Python}. Αποφοίτησε το έτος 2015 από τα Εκπαιδευτήρια Δούκα της ΑΘήνας με γενικό μέσο όρο 16. Τα ενδιαφέροντά του είναι: αθλητισμός και σκάκι. Αποτελεί ενεργό μέλος της ομάδας φοιτητών {\selectlanguage{english} Student Guru} του Πανεπιστημίου Πατρών, που ασχολείται με σύγχρονες τεχνολογίες και εφαρμογές της πληροφορικής. Επίσης, ασχολείται με την αεροδυναμική και μάλιστα αποτελεί μέλος της ομάδας δοκιμών πυραύλων {\selectlanguage{english} Euravia Engineering}.

\end{document}